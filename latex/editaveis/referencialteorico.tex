\chapter[Referencial teórico]{Referencial teórico}

\section{Governo digital}
A partir da transição demográfica e o aproveitamento da democratização do acesso à internet, o governo brasileiro enfrenta desafios oriundos do artigo 6º da Constituição Federal de 1988, de forma a tornar, constantemente, concreta a garantia dos direitos sociais previstos \cite{cristovam_governo_2020}.
Os desafios se apresentam nas adversidades encontradas pelos usuários mais idosos quanto à qualidade da prestação do atendimento digital, tendo em vista o constante aparecimendo desses usuários em atendimentos presenciais.

O Governo Digital pode ser definido na disponibilidade dos serviços administrativos em sua forma digital e remota, consolidando os direitos sociais visando o desenvolvimento e conformidade à hodiernidade da tecnologia.\cite{cristovam_governo_2020}. 
No Brasil, o gov.br é o portal que reúne, em um só lugar, serviços para o cidadão e informações sobre a atuação do Governo Federal. A criação do Portal Governo Digital, no Brasil, vem de 2000, tendo sofrido diversas mudanças tecnológicas e legislativas ao longo dos últimos 23 anos, através de decretos e leis para impulsionar a iniciativa da administração pública em fornecer atendimentos à população. O gov.br foi apresentado em 2019, a fim de unificar e aliar os portais de inclusão digital à marcos civis, políticas de governança digital e sistemas distribuídos de instituiçõe públicas \cite{govbr_2023}.


\section{Idosos e a web}
falar sobre ageismo e sobre acessibilidade voltada para idosos
ansiedade computacional

\section{Usabilidade}
Dentre as definições de usabilidade encontradas na literatura, a maneira mais formal é encontrada na ISO 9241 como “[...] a medida que um produto pode ser usado por usuários específicos, para alcançar objetivos específicos, com eficácia, eficiência e satisfação em um contexto de uso específico” REFERENCIA AQUI. 

Essa área do conhecimento surgiu da necessidade de entendimento sobre o uso de um sistema ou produto, percebido na década de 1980 como um conjunto de fatores que afetam o usuário durante a experiência, se tornando a ligação do indivíduo ao sistema, suas atividades e contexto \cite{weichbroth_usability_2020}. 

\subsection{Apreensibilidade para idosos} 
Por se tratar de uma característica da usabilidade sobre o fator aprendizagem, área regente da pedagogia, sua noção nesse trabalho é guiada para o uso de \emph{software}, sendo o aprendizado voltado para o usuário a aprender no emprego do \emph{software}. Assim como a educação precisa e está extremamente conectada a métodos de avaliação, as definições sobre aprendizagem de uso de \emph{software} é intrínseca às métricas relacionadas para sua análise \cite{grossman_survey_2009}.

Resumidamente, de acordo com Grossman, Fitzmaurice e Attar (2009), a apreensibilidade pode ser definida como a habilidade de atingir uma performance específica em uma tarefa por um usuário com ou sem experiência de sistemas, entretanto é uma definição derivada de uma literatura não muito rica e longe de um consenso \cite{grossman_survey_2009}. Nota-se a variedade e por fim, a subjetividade, nas definições disponíveis e na taxonomia formada pelos autores (COLOCAR A IMAGEM??)

Em idosos, a apreensibilidade pode ser afetada por uma série de fatores, como declínio cognitivo, problemas de visão e audição, e dificuldades na leitura \cite{w3c_older}

É importante levar em conta essas limitações ao projetar materiais de informação e instruções para idosos. Isso pode incluir o uso de fontes de tamanho maior e mais legíveis, evitar jargões técnicos e termos desconhecidos, e fornecer exemplos ilustrativos.

Além disso, é importante considerar a possibilidade de problemas de saúde subjacentes, como demência, que podem afetar a capacidade de um idoso para compreender e seguir instruções. Nesses casos, é importante fornecer suporte adicional, como acompanhamento pessoal ou adaptações no ambiente.

Em resumo, a apreensibilidade para idosos pode ser afetada por uma variedade de fatores, incluindo declínio cognitivo, problemas de visão e audição, dificuldades na leitura e problemas de saúde subjacentes. É importante levar em conta essas limitações ao projetar materiais de informação e instruções para idosos, e fornecer suporte adicional se necessário.

\section{Métodos de avaliação}
elencar os métodos de avaliação que levam em conta os fatores de idade para quantificar/qualificar usabilidade e em especial, a apreensibilidade

